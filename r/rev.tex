% Options for packages loaded elsewhere
\PassOptionsToPackage{unicode}{hyperref}
\PassOptionsToPackage{hyphens}{url}
%
\documentclass[
]{article}
\usepackage{amsmath,amssymb}
\usepackage{lmodern}
\usepackage{iftex}
\ifPDFTeX
  \usepackage[T1]{fontenc}
  \usepackage[utf8]{inputenc}
  \usepackage{textcomp} % provide euro and other symbols
\else % if luatex or xetex
  \usepackage{unicode-math}
  \defaultfontfeatures{Scale=MatchLowercase}
  \defaultfontfeatures[\rmfamily]{Ligatures=TeX,Scale=1}
\fi
% Use upquote if available, for straight quotes in verbatim environments
\IfFileExists{upquote.sty}{\usepackage{upquote}}{}
\IfFileExists{microtype.sty}{% use microtype if available
  \usepackage[]{microtype}
  \UseMicrotypeSet[protrusion]{basicmath} % disable protrusion for tt fonts
}{}
\makeatletter
\@ifundefined{KOMAClassName}{% if non-KOMA class
  \IfFileExists{parskip.sty}{%
    \usepackage{parskip}
  }{% else
    \setlength{\parindent}{0pt}
    \setlength{\parskip}{6pt plus 2pt minus 1pt}}
}{% if KOMA class
  \KOMAoptions{parskip=half}}
\makeatother
\usepackage{xcolor}
\IfFileExists{xurl.sty}{\usepackage{xurl}}{} % add URL line breaks if available
\IfFileExists{bookmark.sty}{\usepackage{bookmark}}{\usepackage{hyperref}}
\hypersetup{
  pdftitle={Estudos em Econometria 2},
  pdfauthor={Felipe Bailez},
  hidelinks,
  pdfcreator={LaTeX via pandoc}}
\urlstyle{same} % disable monospaced font for URLs
\usepackage[margin=1in]{geometry}
\usepackage{color}
\usepackage{fancyvrb}
\newcommand{\VerbBar}{|}
\newcommand{\VERB}{\Verb[commandchars=\\\{\}]}
\DefineVerbatimEnvironment{Highlighting}{Verbatim}{commandchars=\\\{\}}
% Add ',fontsize=\small' for more characters per line
\usepackage{framed}
\definecolor{shadecolor}{RGB}{248,248,248}
\newenvironment{Shaded}{\begin{snugshade}}{\end{snugshade}}
\newcommand{\AlertTok}[1]{\textcolor[rgb]{0.94,0.16,0.16}{#1}}
\newcommand{\AnnotationTok}[1]{\textcolor[rgb]{0.56,0.35,0.01}{\textbf{\textit{#1}}}}
\newcommand{\AttributeTok}[1]{\textcolor[rgb]{0.77,0.63,0.00}{#1}}
\newcommand{\BaseNTok}[1]{\textcolor[rgb]{0.00,0.00,0.81}{#1}}
\newcommand{\BuiltInTok}[1]{#1}
\newcommand{\CharTok}[1]{\textcolor[rgb]{0.31,0.60,0.02}{#1}}
\newcommand{\CommentTok}[1]{\textcolor[rgb]{0.56,0.35,0.01}{\textit{#1}}}
\newcommand{\CommentVarTok}[1]{\textcolor[rgb]{0.56,0.35,0.01}{\textbf{\textit{#1}}}}
\newcommand{\ConstantTok}[1]{\textcolor[rgb]{0.00,0.00,0.00}{#1}}
\newcommand{\ControlFlowTok}[1]{\textcolor[rgb]{0.13,0.29,0.53}{\textbf{#1}}}
\newcommand{\DataTypeTok}[1]{\textcolor[rgb]{0.13,0.29,0.53}{#1}}
\newcommand{\DecValTok}[1]{\textcolor[rgb]{0.00,0.00,0.81}{#1}}
\newcommand{\DocumentationTok}[1]{\textcolor[rgb]{0.56,0.35,0.01}{\textbf{\textit{#1}}}}
\newcommand{\ErrorTok}[1]{\textcolor[rgb]{0.64,0.00,0.00}{\textbf{#1}}}
\newcommand{\ExtensionTok}[1]{#1}
\newcommand{\FloatTok}[1]{\textcolor[rgb]{0.00,0.00,0.81}{#1}}
\newcommand{\FunctionTok}[1]{\textcolor[rgb]{0.00,0.00,0.00}{#1}}
\newcommand{\ImportTok}[1]{#1}
\newcommand{\InformationTok}[1]{\textcolor[rgb]{0.56,0.35,0.01}{\textbf{\textit{#1}}}}
\newcommand{\KeywordTok}[1]{\textcolor[rgb]{0.13,0.29,0.53}{\textbf{#1}}}
\newcommand{\NormalTok}[1]{#1}
\newcommand{\OperatorTok}[1]{\textcolor[rgb]{0.81,0.36,0.00}{\textbf{#1}}}
\newcommand{\OtherTok}[1]{\textcolor[rgb]{0.56,0.35,0.01}{#1}}
\newcommand{\PreprocessorTok}[1]{\textcolor[rgb]{0.56,0.35,0.01}{\textit{#1}}}
\newcommand{\RegionMarkerTok}[1]{#1}
\newcommand{\SpecialCharTok}[1]{\textcolor[rgb]{0.00,0.00,0.00}{#1}}
\newcommand{\SpecialStringTok}[1]{\textcolor[rgb]{0.31,0.60,0.02}{#1}}
\newcommand{\StringTok}[1]{\textcolor[rgb]{0.31,0.60,0.02}{#1}}
\newcommand{\VariableTok}[1]{\textcolor[rgb]{0.00,0.00,0.00}{#1}}
\newcommand{\VerbatimStringTok}[1]{\textcolor[rgb]{0.31,0.60,0.02}{#1}}
\newcommand{\WarningTok}[1]{\textcolor[rgb]{0.56,0.35,0.01}{\textbf{\textit{#1}}}}
\usepackage{graphicx}
\makeatletter
\def\maxwidth{\ifdim\Gin@nat@width>\linewidth\linewidth\else\Gin@nat@width\fi}
\def\maxheight{\ifdim\Gin@nat@height>\textheight\textheight\else\Gin@nat@height\fi}
\makeatother
% Scale images if necessary, so that they will not overflow the page
% margins by default, and it is still possible to overwrite the defaults
% using explicit options in \includegraphics[width, height, ...]{}
\setkeys{Gin}{width=\maxwidth,height=\maxheight,keepaspectratio}
% Set default figure placement to htbp
\makeatletter
\def\fps@figure{htbp}
\makeatother
\setlength{\emergencystretch}{3em} % prevent overfull lines
\providecommand{\tightlist}{%
  \setlength{\itemsep}{0pt}\setlength{\parskip}{0pt}}
\setcounter{secnumdepth}{-\maxdimen} % remove section numbering
\usepackage[brazilian]{babel}
\usepackage[utf8]{inputenc}
\ifLuaTeX
  \usepackage{selnolig}  % disable illegal ligatures
\fi

\title{Estudos em Econometria 2}
\author{Felipe Bailez}
\date{15/06/2021}

\begin{document}
\maketitle

\hypertarget{programa-do-curso}{%
\subsection{Programa do Curso}\label{programa-do-curso}}

\begin{tabular}{ |c|c|c| } 
 \hline
OLS & Cunningham & Cap. 1 e 2\\ 
DAGs & Cunningham & Cap. 3\\ 
Potential Outcomes & Cunningham & Cap. 4\\
Matching & Cunningham & Cap. 1 e 2\\
Discrete Choice Models (Probit/Logit) & Wooldridge & Cap. 17\\ 
Regression Discontinuity & Cunningham & Cap. 6\\
Instrumental Variables & Cunningham & Cap. 7\\
Panel Data & Cunningham & Cap. 7\\
DiD & Cunningham & Cap. 9\\
Two-way FE with differential timing & Cunningham & Cap. 9\\ 
 \hline

\end{tabular}
\newpage

\hypertarget{ols}{%
\subsection{OLS}\label{ols}}

Estimando a OLS através de matrizes (Wooldrigde Appendix E):
\emph{\[ y_t = \beta_1 + \beta_2 x_{t2} + \beta_3 x_{t3} + ... + \beta_k x_{tk} + u_t 
, t = 1,2,...,n\]}

Para cada \(t\), definimos um vetor \(1 \times k\) onde \emph{\$ x\_t =
(1,x\_2,\ldots,x\_\{tk\}) \$ e
\(\beta = (\beta_1, \beta_2, ..., \beta_k)'\)}, então

\[ y_t =  x_t \beta + u_t , t = 1,2, ... , n\]

\[ y = X \beta + u \]

Para calcular os coeficientes:

\[ X'(y - X \hat{\beta} )= 0\] \[ (X'X)\hat{\beta} = X'y \]
\[ \hat{\beta} = (X'X)^{-1} X'y \]

\hypertarget{exemplo-em-r}{%
\subsubsection{Exemplo em R}\label{exemplo-em-r}}

\begin{Shaded}
\begin{Highlighting}[]
\NormalTok{Y }\OtherTok{\textless{}{-}} \FunctionTok{matrix}\NormalTok{(}\FunctionTok{c}\NormalTok{(}\FloatTok{1.5}\NormalTok{,}\FloatTok{6.5}\NormalTok{,}\DecValTok{10}\NormalTok{,}\DecValTok{11}\NormalTok{,}\FloatTok{11.5}\NormalTok{,}\FloatTok{16.5}\NormalTok{), }\AttributeTok{ncol=}\DecValTok{1}\NormalTok{, }\AttributeTok{nrow=}\DecValTok{6}\NormalTok{, }\AttributeTok{byrow=}\NormalTok{F)}

\NormalTok{X }\OtherTok{\textless{}{-}} \FunctionTok{matrix}\NormalTok{(}\FunctionTok{c}\NormalTok{(}\DecValTok{1}\NormalTok{,}\DecValTok{1}\NormalTok{,}\DecValTok{1}\NormalTok{,}\DecValTok{1}\NormalTok{,}\DecValTok{1}\NormalTok{,}\DecValTok{1}\NormalTok{,}\DecValTok{0}\NormalTok{,}\DecValTok{1}\NormalTok{,}\DecValTok{1}\NormalTok{,}\DecValTok{2}\NormalTok{,}\DecValTok{2}\NormalTok{,}\DecValTok{3}\NormalTok{,}\DecValTok{0}\NormalTok{,}\DecValTok{2}\NormalTok{,}\DecValTok{4}\NormalTok{,}\DecValTok{2}\NormalTok{,}\DecValTok{4}\NormalTok{,}\DecValTok{6}\NormalTok{), }\AttributeTok{ncol =} \DecValTok{3}\NormalTok{, }\AttributeTok{nrow =} \DecValTok{6}\NormalTok{, }\AttributeTok{byrow=}\NormalTok{F)}

\NormalTok{Xt }\OtherTok{=} \FunctionTok{t}\NormalTok{(X)}
\NormalTok{XtX }\OtherTok{=}\NormalTok{ Xt }\SpecialCharTok{\%*\%}\NormalTok{ X}
\NormalTok{XtX\_inv }\OtherTok{=} \FunctionTok{solve}\NormalTok{(XtX)}
\NormalTok{Xty }\OtherTok{=}\NormalTok{ Xt }\SpecialCharTok{\%*\%}\NormalTok{ Y}

\NormalTok{beta\_hat }\OtherTok{=}\NormalTok{ XtX\_inv }\SpecialCharTok{\%*\%}\NormalTok{ Xty}
\end{Highlighting}
\end{Shaded}

\newpage

\hypertarget{potential-outcomes}{%
\subsection{Potential Outcomes}\label{potential-outcomes}}

\begin{Shaded}
\begin{Highlighting}[]
\CommentTok{\# Algum codigo aqui}
\end{Highlighting}
\end{Shaded}

\newpage

\hypertarget{matching}{%
\subsection{Matching}\label{matching}}

\begin{Shaded}
\begin{Highlighting}[]
\CommentTok{\# Algum codigo aqui}
\end{Highlighting}
\end{Shaded}

\newpage

\hypertarget{discrete-choice-models}{%
\subsection{Discrete Choice Models}\label{discrete-choice-models}}

\begin{Shaded}
\begin{Highlighting}[]
\CommentTok{\# Algum codigo aqui}
\end{Highlighting}
\end{Shaded}

\newpage

\hypertarget{regression-discontinuity}{%
\subsection{Regression Discontinuity}\label{regression-discontinuity}}

\begin{Shaded}
\begin{Highlighting}[]
\CommentTok{\# Algum codigo aqui}
\end{Highlighting}
\end{Shaded}

\newpage

\hypertarget{instrumental-variables}{%
\subsection{Instrumental Variables}\label{instrumental-variables}}

\hypertarget{homogenous-treatment-effects-cunningham-cap.-7}{%
\subsubsection{\texorpdfstring{\emph{Homogenous Treatment Effects
(Cunningham Cap.
7)}}{Homogenous Treatment Effects (Cunningham Cap. 7)}}\label{homogenous-treatment-effects-cunningham-cap.-7}}

Em \emph{Homogenous Treatment Effects}, você supõe que todas as pessoas
do grupo que receberam o tratamento terão uma mudança na variável de
interesse com a mesma intensidade. Ou seja, se fazer universidade
aumento minha renda em 10\%, então aumentou em 10\% para todos que
fizeram universidade.

Portanto, suponha um modelo onde você deseja estimar o quanto um aumento
de educação causa aumento na renda.

\[ Y_i = \alpha + \delta S_i + \gamma A_i + \varepsilon_i\]

Onde, \(Y_i\) é a renda de cada individuo, \(S_i\) os anos de educação e
\(A_i\) uma variável não observada que representa abilidade. Desse modo,
o modelo que conseguiremos estimar é o seguinte:

\[ Y_i = \alpha + \delta S_i + \eta_i\]

onde \(\eta_i\) é o erro composto equivalente a
\(\gamma A_i + \varepsilon_i\). Como assumimos que ``abilidade'' está
correlacionada com a variável de ``educação'', então apenas
\(\varepsilon_i\) está descorrelacionado com os regressores.

utilizando o valor estimado de \(\hat{\delta}\) da OLS tradicional temos
que

\[ \hat{\delta} =  \frac{Cov(Y,S)}{Var(S)} = \frac{E[YS]- E[Y] E[S]}{Var(S)}\]

Se utilizarmos o valor de \(Y\) da regressão onde \(A\) é observável
teremos que

\[ \hat{\delta} =  \frac{E[S (\alpha +\delta S + \gamma A + \epsilon)] - E[\alpha + \delta S + \gamma A + \epsilon] E[S]}{Var(S)}\]
\[ \hat{\delta} =  \frac{\delta E(S^2) - \delta E(S)^2 + \gamma E(AS) - \gamma E(S) E (A) + E(\varepsilon S)+ E(S) E(\varepsilon)}{Var(S)}\]

\[ \hat{\delta} =  \delta + \gamma\frac{Cov(A,S)}{Var(S)}\] Logo, se
\(\gamma > 0\) e \(Cov(A,S) > 0\), então \(\hat{\delta}\) será viesado
para cima.E como deve ser positivamente correlacionada com eduação,
então isso é o que deve acontecer.

Mas se encontrarmos uma nova variável \(Z_i\) que causa as pessoas a ter
mais anos de estudos e que é descorrelacionada com abilidade (as
variáveis não observáveis), podemos utilizar ela como uma variável
instrumental para estimar \(\delta\).

Para isso precisamos primeiro calcular a covariancia de \(Y\) e \(Z\)

\[ Cov(Y,Z) = Cov( \alpha + \delta S + \gamma A + \varepsilon, Z)\]

\[ = E[ Z (\alpha + \delta S + \gamma A + \varepsilon )] - E[\alpha + \delta S + \gamma A + \varepsilon ] \ \ E[Z]\]
\[ =  E[\alpha Z + \delta S Z + \gamma A Z + \varepsilon Z ] -  \ \{ \alpha + \delta E(S) + \gamma E(A) + E(\varepsilon)\ \} \ E[Z]\]

\[ =   \{ \alpha E(Z) + \delta E(SZ) + \gamma E (AZ) + E (\varepsilon Z) \}  -\\
 \{\alpha E(Z) + \delta E(S) E(Z) + \gamma E(A) E(Z) + E(\varepsilon) E(Z) \ \} \]

\[ =   \{ \ \alpha E(Z) - \alpha E(Z) \ \} + \delta \{ \ E (SZ) - E(S) \ E(Z) \ \} + \\
\gamma \ \{  E (AZ) - E(A) \ E(Z) \ \} + \\
\{  E (\varepsilon Z) -  E(\varepsilon) \ E(Z) \}\]

\[ =   \delta \ Cov(S,Z) + \gamma \ Cov(A, Z) + Cov \ (\varepsilon, Z)\]

Como sabemos que \(Cov(A,Z) = 0\) e \(Cov(\varepsilon, Z) = 0\), uma vez
que não existe essa relação entre os instrumentos, podemos estimar
\(\hat{\delta}\).

\[ \hat{\delta} = \frac{Cov(Y,Z)}{Cov(S,Z)} \] Dessa forma, podemos usar
a variável instrumental \(Z\) para estimar \(\hat{\delta}\) caso \(Z\)
seja independente da variável oculta e do erro estrutural da regressão.
Ou seja, o instrumento deve ser independente das duas partes do erro
composto \(\eta_i\) citado no início.

\begin{Shaded}
\begin{Highlighting}[]
\NormalTok{Y }\OtherTok{\textless{}{-}} \FunctionTok{matrix}\NormalTok{(}\FunctionTok{c}\NormalTok{(}\FloatTok{1.5}\NormalTok{,}\FloatTok{6.5}\NormalTok{,}\DecValTok{10}\NormalTok{,}\DecValTok{11}\NormalTok{,}\FloatTok{11.5}\NormalTok{,}\FloatTok{16.5}\NormalTok{), }\AttributeTok{ncol=}\DecValTok{1}\NormalTok{, }\AttributeTok{nrow=}\DecValTok{6}\NormalTok{, }\AttributeTok{byrow=}\NormalTok{F)}
\NormalTok{X }\OtherTok{\textless{}{-}} \FunctionTok{matrix}\NormalTok{(}\FunctionTok{c}\NormalTok{(}\DecValTok{1}\NormalTok{,}\DecValTok{1}\NormalTok{,}\DecValTok{1}\NormalTok{,}\DecValTok{1}\NormalTok{,}\DecValTok{1}\NormalTok{,}\DecValTok{1}\NormalTok{,}\DecValTok{0}\NormalTok{,}\DecValTok{1}\NormalTok{,}\DecValTok{1}\NormalTok{,}\DecValTok{2}\NormalTok{,}\DecValTok{2}\NormalTok{,}\DecValTok{3}\NormalTok{,}\DecValTok{0}\NormalTok{,}\DecValTok{2}\NormalTok{,}\DecValTok{4}\NormalTok{,}\DecValTok{2}\NormalTok{,}\DecValTok{4}\NormalTok{,}\DecValTok{6}\NormalTok{), }\AttributeTok{ncol =} \DecValTok{3}\NormalTok{, }\AttributeTok{nrow =} \DecValTok{6}\NormalTok{, }\AttributeTok{byrow=}\NormalTok{F)}
\end{Highlighting}
\end{Shaded}

\hypertarget{two-stage-least-squares}{%
\subsubsection{\texorpdfstring{\emph{Two-stage least
squares}}{Two-stage least squares}}\label{two-stage-least-squares}}

Uma forma mais intuitiva de trabalhar com Variáveis Instrumentais é
através das \emph{Two-stage least squares} (ou \(2SLS\)). Seguindo o
raciocínio de antes, suponha que temos dados de \(Y\), \(S\) e \(Z\)
para cada observação \(i\). Nesse caso o processo de geração de dados é
dado por:

\[ Y_i = \alpha + \delta S_i + \varepsilon_i\]

\[ S_i = \gamma + \beta Z_i + \epsilon_i\]

onde \$ Cov(Z,\varepsilon) = 0 \$ e \$\beta \neq 0 \$ . Sabendo que
\(\sum_{i=1}^n (X-i \bar{x}) = 0\), podemos reescrever o estimador de
variável instrumental como

\[ \hat{\delta} = \frac{Cov(Y,Z)}{Cov(S,Z)}\]
\[ = \frac{ \frac{1}{n} \sum_{i=1}^n (Z_i - \bar{Z} ) (Y_i - \bar{Y}) } { \frac {1} {n} \sum_{i=1}^n (Z_i - \bar{Z}) (S_i - \bar{S}) }\]
\[ = \frac{ \frac{1}{n} \sum  } {\frac{1}{n} } \]

\newpage

\hypertarget{panel-data}{%
\subsection{Panel Data}\label{panel-data}}

\begin{Shaded}
\begin{Highlighting}[]
\CommentTok{\# Algum codigo aqui}
\end{Highlighting}
\end{Shaded}

\newpage

\hypertarget{diff-in-diff}{%
\subsection{Diff-in-Diff}\label{diff-in-diff}}

\begin{Shaded}
\begin{Highlighting}[]
\CommentTok{\# Algum codigo aqui}
\end{Highlighting}
\end{Shaded}

\newpage

\hypertarget{two-way-fixed-effects-with-differential-timing}{%
\subsection{Two-way Fixed-Effects with differential
timing}\label{two-way-fixed-effects-with-differential-timing}}

\begin{Shaded}
\begin{Highlighting}[]
\CommentTok{\# Algum codigo aqui}
\end{Highlighting}
\end{Shaded}


\end{document}
